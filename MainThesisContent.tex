\documentclass[bp,en]{FEIstyle}


\usepackage{ amssymb }
\usepackage{float}



\FEIauthor{Erik Ziman}
\FEItitle{Kryptografia na báze eliptických kriviek - analýza bezpečnosti a demonštrácia útokov}
\FEItitleEn{Elliptic curve cryptography - security analysis and attack demonstration}
\FEIregNr{FEI-UNKNOWN}
\FEIsupervisor{Mgr. Karina Chudá, PhD.}
%\FEIconsultant{Ing. John Doe}

\FEIkeywords{kľúčové slovo1, kľúčové slovo2, kľúčové slovo3}
\FEIkeywordsEn{keyword1, keyword2, keyword3}
% \FEIglossaries{includes/glossary}
\bibliography{includes/bibliography.bib}

\iffalse \newacronym{sw}{SW}{Star Wars}
\newacronym{hw}{HW}{Halo Wars}
\newacronym{cdma}{CDMA}{Code Division Multiple Access} 
\newacronym{gsm}{GSM}{Global System for Mobile communication}
 \fi



\newcommand{\point}[1]{
#1 = (x_{#1}, y_{#1})
}

\newcommand{\ycor}[1]{
y_{P_{#1}}
}

\newcommand{\xcor}[1]{
x_{P_{#1}}
}

\begin{document}
As Yale University professor Serge Lang once noted in the beginning of his book \textit{Elliptic Curves: Diophantine Analysis}, ``It is possible to write endlessly on elliptic curves. (This is not a threat)." Indeed, elliptic curves form a deep and intricate subject. In this thesis, we will dive into the topic of elliptic curves, with a particular focus on their impact in the field of modern cryptography. We aim to explore their significance and key properties, highlighting the crucial role they play in many cryptographic systems. The primary aim of this thesis is to deepen our understanding of this complex topic to such an extent that we are able to implement elliptic curve cryptography (ECC) in various applications and programs. In addition to exploring the theoretical significance of elliptic curves, we will also implement a few examples of elliptic curves referenced with corresponding source code and visual representation. Furthermore, this thesis will shed light on various attacks and common mistakes in the implementation of these curves, emphasizing the importance of secure and correct practices when building cryptographic applications.
\section*{Motivation for Using Elliptic Curves}
The main advantage of ECC is the degree of security it provides when considering its comparatively smaller key sizes.

\begin{table}[h]
    \centering
    \caption{Time to Break vs Key Sizes}
    \begin{tabular}{|c|c|c|}
        \hline
        \textbf{Time to Break (MIPS-years)} & \textbf{ECC Key Size (bits)} & \textbf{RSA Key Size (bits)} \\
        \hline
        $10^4$  & 106   & 512  \\ \hline
        1$0^8$  & 132   & 768  \\ \hline
        $10^{11}$    & 160   & 1024   \\ \hline
        $10^{20}$    & 210   & 2048   \\ \hline
        $10^{78}$    & 600   & 21000  \\
        \hline
    \end{tabular}
\end{table}


With the keys being smaller, we are able to have better computational efficiency of the algorithms; thus, our requirements on hardware resources can lower. Another advantage is that almost all currently known systems based on the discrete logarithm problem can be converted into elliptic curve-based systems. The vast majority of elliptic curve cryptography schemes rely on the Elliptic Curve Discrete Logarithm Problem (ECDLP) for their security.
\section*{What is an Elliptic Curve?}
In order to start defining elliptic curves, we first need to introduce some terms up front.
\subsection*{Algebraic curve:}
An \textit{algebraic curve} (over field $K$) is set of points (x, y) in the plane that satisfy a a non-constant polynomial equation in two variables. A \textit{nonsingular algebraic curve} is algebraic curve without any singular points.
\[
A: = \{ (x,y) \in K^2 \mid f(x, y) = 0\}  
\]
% \subsection*{K-rational point:}
% A $K$-\textit{rational point} is a solution $(x, y)$ to the equation of an algebraic curve, where both $x$ and $y$ are in the field $K$.
\subsection*{K-rational point:}
A \( K \)-\textit{rational point} is a solution \( (x, y) \) to the equation of an algebraic curve, where both \( x \) and \( y \) are in the field \( K \). 

\[
\point{P} \quad \text{where } f(P) = 0 \text{ with } x_{P}, y_{P} \in K.
\]

\subsection*{Point at infinity:}
The point at infinity $\mathcal{O}$ is the identity element of elliptic curve arithmetic. Adding this point to any other point (including itself) results in that other point:
\[
\mathcal{O} + P = P 
\]
\[
\mathcal{O} + \mathcal{O} = \mathcal{O} 
\]
\subsection*{Elliptic curve:}
An \textit{elliptic curve} (over field $K$) is a nonsingular cubic curve, with at least 1 $K$-rational point. We will primarily be working with curves that are defined by The Weierstrass Form:
\[
E := \{ (x,y) \in K^2 \mid y^2 = x^3 + ax + b \} \cup \{ \mathcal{O} 
\} \quad \text{with } a, b \in K
\]
Elliptic curve is considered to form a group if it's cubic polynomial, has no repeated roots.
\newpage

\subsection*{Examples over $\mathbb{R}$:}
\begin{figure}[H]
    \centering
    \begin{subfigure}{0.3\textwidth}
        \includegraphics[width=\linewidth]{img/curve__y^2_=_x^3_+4.png}
        \caption{$y^2 = x^3 + 4$}
        \label{fig:example_curve_1}
    \end{subfigure}%
    \hfill % Add horizontal space between figures
    \begin{subfigure}{0.3\textwidth}
        \includegraphics[width=\linewidth]{img/curve__ y^2_=_x^3_-2x_+_3.png}
        \caption{$y^2 = x^3 - 2x + 3$}
        \label{fig:example_curve_2}
    \end{subfigure}%
    \hfill % Add horizontal space between figures
    \begin{subfigure}{0.3\textwidth}
        \includegraphics[width=\linewidth]{img/curve__y^2_=_x^3_-5x_+2.png}
        \caption{$y^2 = x^3 - 5x + 2$}
        \label{fig:example_curve_3}
    \end{subfigure}
    \caption{Elliptic curves (over $\mathbb{R}$)}
    \label{fig:example_elliptic_curves_1}
\end{figure}

\begin{figure}[H]
    \centering
    \begin{subfigure}{0.5\textwidth}
        \includegraphics[width=\linewidth]{img/curve_finite_y2=x3-x+3.png}
        \caption{$y^2=x^3-x+3 \mod 97$}
        \label{fig:example_curve_4}
    \end{subfigure}%
    \begin{subfigure}{0.4841\textwidth}
        \includegraphics[width=\linewidth]{img/curve_finite-y2=x3-x-3.png}
        \caption{$y^2=x^3-x-3 \mod 47$}
        \label{fig:example_curve_5}
    \end{subfigure}%
    \caption{Elliptic curves (over finite fields)}
    \label{fig:example_elliptic_curves_2}
\end{figure}
% First, we are going to give a few examples over the field $\mathbb{R}$, then (for reasons that will be explained later), we are going to focus more on analyzing curves over finite fields.
\newpage
\section*{Operations on Elliptic Curves}
Now that we know what an elliptic curve is, let's define the operational rules for performing point calculations on these curves. \\
$E := \{ (x,y) \in K^2 \mid y^2 = x^3 + ax + b \} \cup \{ \mathcal{O} 
\} \quad \text{with } a, b \in K$ \\ 
$\point{P_i}$ with $P_i \in E$
\subsection*{Negating a point:}
\[ 
P = (x_P, y_P) 
\]
\[ 
-P = -(x_P, y_P) = (x_P, - y_P)
\]
\begin{figure}[H]
    \centering
    \begin{subfigure}{0.4\textwidth}
        \includegraphics[width=\linewidth]{img/negating_point.png}
        \label{fig:negating_point_curve_1}
    \end{subfigure}%
    \begin{subfigure}{0.4255\textwidth}
        \includegraphics[width=\linewidth]{img/negating_point_finite.png}
        \label{fig:negating_point_curve_2}
    \end{subfigure}%
    \caption{Visual representation of point negation }
    \label{fig:negating_point_curves}
\end{figure}
% Notice how $P$ and $-P$ are inverse to each other.
\subsection*{Finding inverse of a point:}
\[ 
P = (x_P, y_P) 
\]
\[ 
P^{-1} = -P 
\]
\subsection*{Addition:}
Addition is commutative meaning $P_i + P_j = P_j + P_i$. \\
In these examples below we asume that $P_3 = P_1 + P_2$\\
\text{ 1.) if } $P_1 \neq P_2$ and $P_1,P_2 \neq \mathcal{O} $:
\[
\lambda = \frac{y_{P_2} - y_{P_1}}{x_{P_2} - x_{P_1}}
\]
\[
P_3 = (\lambda^2 - x_{P_1} - x_{P_2}\text{ , } \lambda(x_{P_1} - x_{P_3}) - y_{P_1})
\]
\begin{figure}[H]
    \centering
    \begin{subfigure}{0.38\textwidth}
        \includegraphics[width=\linewidth]{img/addition_ecc.png}
        \label{fig:addition_curve_1}
    \end{subfigure}%
    \begin{subfigure}{0.412\textwidth}
    
        \includegraphics[width=\linewidth]{img/finite_addition.png}
        \label{fig:addition_curve_2}
    \end{subfigure}%
    \caption{Visual representation of point addition (case 1)}
    \label{fig:addition_elliptic_curves_1}
\end{figure}


\text{2.) if } $P_1 = P_2$  and $y_{P_1},\ycor{2} \neq 0$ and $P_1,P_2 \neq \mathcal{O} $: \par
\[
m = \frac{3x_{P_1}^2 + a}{2y_{P_1}}
\]
\[
P_3 = (m^2 - 2{x_{P1}}\text{ , } m(x_{P_1} - x_{P_3}) - y_{P_1})
\]
\begin{figure}[H]
    \centering
    \begin{subfigure}{0.38\textwidth}
        \includegraphics[width=\linewidth]{img/addition_tangent.png}
        \label{fig:addition_curve_3}
    \end{subfigure}%
    \begin{subfigure}{0.42\textwidth}
        \includegraphics[width=\linewidth]{img/tangent_finite_last.png}
        \label{fig:addition_curve_4}
    \end{subfigure}%
    \caption{Visual representation of point addition (case 2)}
    \label{fig:addition_elliptic_curves_2}
\end{figure}


\text{3.) if } $P_1 = P_2$  and $\ycor{1},\ycor{2} = 0$ \par
\[
P_3 = \mathcal{O} 
\]
\begin{figure}[ht]
    \centering
    \begin{subfigure}{0.4\textwidth}
        \includegraphics[width=\linewidth]{img/addition_same_point_2.png}
        \label{fig:addition_curve_5}
    \end{subfigure}%
    \begin{subfigure}{0.405\textwidth} 
        \includegraphics[width=\linewidth]{img/infinity_addition.png}
        \label{fig:addition_curve_6}
    \end{subfigure}%
    \caption{Visual representation of point addition (case 3)}
    \label{fig:addition_elliptic_curves_3}
\end{figure}
\text{4.) if } $P_1 \neq \mathcal{O}$ , $P_2 = \mathcal{O}$  \par
\[
P_3 = P_1 
\]

\text{5.) if } $P_1 = -P_2$ 
\[
P_3 = \mathcal{O}
\]
\subsection*{Subtraction:}
\[
P_1 - P_2 = P_1 + (-P_2)
\]
\[ 
(x_{P_1}, y_{P_1}) - (x_{P_2}, y_{P_2}) = (x_{P_1}, y_{P_1}) + (x_{P_2}, -y_{P_2})
\]
\subsection*{Multiplication:}
Only scalar multiplication is possible. By multiplication, we understand repeated addition of point to itself.
\[
k\times P = P + P + ... \text{ } k \text{ times }
\] % \textit{Note:} For slope calculation use the formula when $P = Q$.
% \subsection*{Division:}
% Only scalar division is possible. 
% \[
% \frac{1}{a} (x_p, y_p) = a^{-1} (x_p, y_p)
% \]



\subsubsection*{Larger multiples of points:}
This works well in theory but what if $k$ was a really large number? It is obvious that in order to implement secure elliptic curve based algorithms, we will need to work with big multiples of points. The faster we can get to the result, the better. There is a number of techniques which can help us achieve faster computation of these big point multiplications.


\subsubsection*{Double and add method:}
We already know that elliptic curves form a group over finite field $\mathbb{F}_\mathbb{P}$ considering $P$ is a non-even prime. This means that whenever we add any of the two members of this group together the result will also have to be a member of this group. For example:
\[
3 \times P + 9 \times P = 12 \times P
\]
As we saw in this example (TODO add reference to tangent point addition) adding a point $P_1$ to $P_2$ is being calculated the same way as adding $P_1$ to itself (assuming $P_1=P_2$). Since $P_1+P_1=2\times P_1$, adding a point to itself is the same operation as doubling the point. Now we have effective way for doubling a point using simple addition. 

We can leverage this by taking $k_2$ and start progressively doubling P as many times as there are binary digits from least significant bit (LSB) up to most significant "1" bit. For each "1" bit in $k$'s binary form, we add the corresponding multiple of 2 $\times$ P to the accumulated result. Here's an example:

\[
41 \times P 
\]
\[
41_{10} = 110011_2  
\]
\begin{table}[H]
\centering
\begin{tabular}{|c|c|c|}
\hline
\textbf{Bits of 41} & \textbf{Current Doubling} & \textbf{Result (After Addition if bit = 1)} \\ \hline
1 & $P$ & $P$ \\ \hline
1 & $2\times P$ & $P+2\times P=3\times P$ \\ \hline
0 & $4\times P$ & $3\times P$ \\ \hline
0 & $8\times P$ & $3\times P$ \\ \hline
1 & $16\times P$ & $3\times P+16\times P=19\times P$ \\ \hline
1 & $32\times P$ & $19\times P+32\times P=41\times P$ \\ \hline
\end{tabular}
\caption{Binary Representation and Double-and-Add Method for $41 \times P$}\end{table}




This computation uses $log_2(n)$ multiplications and on average $\frac{1}{2}log_2(n)$ additions. 


\newpage

\begin{algorithm}
\caption{Double and add algorithm for point multiplication }  
\label{alg:double-and-add-safer}                         
\begin{algorithmic}[1]                    
    \STATE $P_3 = \mathcal{O}$
    \STATE $P_2 = P_1$ 
    \WHILE{$k > 0$}
        \IF{$k \bmod 2 = 1$} 
            \STATE $P_3= P_3 + P_2$ 
        \ENDIF
        \STATE $P_2 = 2 \times P_2$ 
        \STATE $k >> 1$
    \ENDWHILE
    \RETURN $P_3$
\end{algorithmic}
\end{algorithm}



TODO : ADD TERNARY REPRESENTATION OF BINARY DIGITS
for example 127 in binary is 111111 but in "ternary" it would be 100000-1

If we don't limit ourselves to only binary representation of a number $k$ we are able to optimize the computing time and resources even further by introducing set of coefficients from {1,0} to {-1,0,1}. What this will do is we will 
1. have to subtract some points multiples 
2. have less additions to compute (which will save our time and resources)

Subtraction is of same complexity as addition since it is exactly the same operation with a twist of negating y coorginate. 

\newpage

\section*{EC in key exchange mechanisms:}

\subsection*{Discrete logarithm problem DLP:}

DLP is a "one-way" problem area in mathematics which considers the following qualities of modular arithmetic combined with exponential functions:\\
- It relatively easy to compute $a^b \mod{p}$ when given a, b and p\\
- However, finding b when given a and p is much harder task\\ 
This asymmetry is fundemental DLP concept and it is one of the key 
elements which many cryptographic protocols use when relying on complexity of this problem.


\subsection*{Diffie-Hellman key exchange DH:}

TODO : add this 


\subsection*{Elliptic Curve Discrete Logarithm Problem ECDLP:}

TODO : add this 

\subsection*{Elliptic Curve Diffie-Hellman key exchange ECDH:}

TODO : add this 


\newpage

% \section*{Specific Examples}

% TODO : consider where (or if at all) to include this section \par
% \[
% E: = \{ (x,y) \in \mathbb{R}^2 \mid y^2=x^3-x+1\}\cup\{\mathcal{O}\} 
% \]
% \[
% P_1 = (1, 1), P_2 = (0, -1), P_3 = ?
% \]
\end{document}